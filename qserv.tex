\subsection{Case Study: Data Access Services}

The Data Access Services are a subset of the online services provided by the LSST Science Platform, providing
access to catalog and image data within the LSST data releases as well as some management of user-generated
data products.  In general the services are multi-user, scalable, and designed to be deployed in a data
center context rather than on individual developer machines.

The various components of the Data Access Services are built within LSST CI (Jenkins) as Docker containers.
Per general LSST practice, unit tests included in each module are executed as part of the build, and the build
will be failed if these do not pass.  In addition, some parts of the service suite have coverage by small-
scale automated integration tests; these tests are run by Travis CI on each GitHub branch commit or pull
request.  The automated integration tests launch a constellation of containers configured with test datasets,
then make a series of service requests and evaluate received responses against known/expected results.

The Data Access Services are at present deployed at scale for testing and development on three compute
clusters: one thirty node cluster at NCSA, and two twenty-five node clusters at CC-IN2P3.  These deployments
host a combination of synthetic test data and scientifically valid test data sourced from other astronomical
surveys such as SDSS, WISE, and HSC.  Scale testing is re-provisioned annually on clusters of increasing size
with datasets of increasing size, on a ramp toward the scale necessary to support start of operations.

Cluster deployments of the Data Access Services containers are done at present with a collection of ad-hoc
administration scripts, but the project is working currently to replace these with Kubernetes tooling.

The Data Access services have been incubated along with the rest of the LSST software stack, sharing common CI
infrastructure, software package management, build tooling, and release cadences.  As the project has
progressed, however, this tight coupling has become somewhat taxing.  For example, the Data Access services
and the Data Release Production codes have a large set of disjoint software dependencies, but shared
packaging, CI, and release tooling result in an overlapping package dependency graph which causes developers
in both groups to commonly build codes for which they have little need in their day-to-day work. Additionally,
while much of the LSST software stack prioritizes platform and toolchain portability in order to foster
individual collaborations, the return for effort invested on this is probably lower for datacenter scale
services delivered as fully portable containers.  Lastly, as the project moves into commissioning and
operations, it seems likely that periods of heavy activity on the data production and data service codes will
phase and there will be a desire for decoupled release cadences for these.  In light of these concerns the
project is currently working to decouple Data Access Services CI, toochain/platform support requirements,
and and release cadences from the rest of the LSST software stack, while maintaining common code quality
standards and development practices.
