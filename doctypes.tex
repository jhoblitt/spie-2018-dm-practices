\subsection{Types of Documentation}
\label{sec:doc_types}

Data Management creates a variety of documentation for different audiences, and with different purposes.
Broadly, we break documentation efforts into two categories: project documentation and user documentation.
Project documentation is a tree of requirements documents, interface control documents, design documents, and test reports that aides the construction project.
User documentation, on the other hand, is a deliverable that we ship alongside our software and services to either the scientific community or future developers and operators.

\subsubsection{Project documentation}
\label{sec:project_docs}

\paragraph{Documentation tree}

DM creates project documentation that fits within LSST's overall Document Management Plan.\cite{LPM-51}
As is standard practice, we organize the collection of project documents in a tree.
LSST Project Management and Systems Engineering requirements documents form the tree's trunk, along with interface control documents that DM collaborates on with other LSST subsystems.
These flow into Data Management requirements documents for each DM product: Database,\cite{LDM-555} Middleware,\cite{LDM-556} Science Platform,\cite{LDM-554} and Data Management System Design.\cite{LDM-148}
From these, we maintain design documents: Database Design,\cite{LDM-135} Middleware Design,\cite{LDM-152} Science Platform Design,\cite{LDM-542} Services and Infrastructure Design,\cite{LDM-129} Pipeline Design,\cite{LDM-151} and Network Design.\cite{LSE-78}
Then from each design document, we maintain test specification documents.
Overall, the DM Validation and Test Plan\cite{LDM-503} coordinates these test specifications.
Finally, we write test reports that implement the test plans.

We have processes for managing project documents and revisions.
The DM Change Control Board (DM-CCB) reviews documents and revisions before they become the accepted baseline.
The DM-CCB also reviews test reports and approves those that specifically refer to formal milestones in the LDM-503 verification and test plan.

\paragraph{Technical notes}

We recognized that DM team members needed to write documents as a means of communication within the project.
These documents might be design proposals, summaries of data processing experiments, or descriptions of a new internal tool or service.
This content does not fit within the formal document tree, nor does it require oversight from the DM-CCB.
Technical notes fit this niche for DM.

Since the defining characteristic of technical notes is the editorial independence of their authors, we sought to automate as much of the maintenance surrounding these documents as possible.
By allowing an author to create a new document and publishing it without either seeking permission or waiting on a human coordinator, we maximize the likelihood that DM team members will choose to share information through technical notes.
Thus we have implemented a self-service system based on ChatOps that allows any Data Management staff member to create and publish a technical note (\ref{sec:chatops}).
These technical notes are edited on GitHub and published as websites, and in the future, we plan to make these documents readily citeable with DOIs and listings in archives.

As a cultural benefit, we see technical notes as a way for junior members to gain exposure within the project, and beyond, by publishing a citeable article featuring their contributions.

\paragraph{Developer Guide}

The DM team is large and distributed, yet we need to work coherently.
An effective way to promote this is through strong documentation of internal processes and policies.
These range from a description of how we communicate, to the development workflow, code style guides, and user guides for internal services.
While we could conceivably publish this information as a kind of project documentation (see above), the content of the Developer Guide doesn't fit into a narrative format.
Instead, the content is made up of isolated, yet inter-linked, how-to pages and guides.
Thus we chose to publish our Developer Guide as a website,\cite{devguide} using the same systems that support our user documentation.
DM team members are encouraged to contribute improvements to the Developer Guide through GitHub.
We use GitHub pull requests to get feedback on contributions and provide approval for changes to controlled pages such as coding style guides.

\subsubsection{User documentation}
\label{sec:user_docs}

User documentation describes what is built, and most importantly, how to operate it.
Thus we consider user documentation to be a deliverable product of the LSST construction project, on par with the software itself.

At the time of this publication, most DM effort towards user documentation have gone into the LSST Science Pipelines product\cite{pipelines-guide} and user guides for internal tooling.
The LSST Science Platform is the second major documentation project.
DM will also lay the groundwork for documentation of both LSST's data releases and prompt products.

We begin a new user documentation project by studying the nature of the product user needs.
Then we work towards an information architecture plan that maps out topics in the user documentation project.
As described by \citenum{Baker:2013}, topics are self-contained (though highly interlinked) articles that conform to a type.
The general categories of topic types are concept guides, tutorials, how-to guides, and references.\cite{Procida:2017}
This exercise not only allows us to systematically map the content needed to document a project but also understand the infrastructure needed to support each topic type (such as the tools needed to generate an API reference, or the tools needed to deliver and test a tutorial).
The Science Pipelines Documentation Design\cite{DMTN-030} technical note is our living design for the LSST Science Pipelines user documentation.
