\subsection{Community forum}
\label{sec:forum}

Data Management operates the LSST Community forum,\footnote{\url{https://community.lsst.org}} which is open to the public.
Although Slack carries the bulk of our team communication because it is fast and team-focused, the Community forum fills an important communications niche by having the opposite qualities.
The forum encourages longer, thoughtful posts that lead to slower-paced conversations.
These slower-paced conversations tend to be more accessible for those who cannot work synchronously the DM team because of factors like time zones or other obligations.
Much of DM's own use of the forum, therefore tends to be intentionally public facing.
For example, we regularly post small announcements of new features in the LSST software.
We also host a support category to crowd-source help for astronomers who are using LSST software in their own investigations.
To reach the LSST user community who are more interested in planning for LSST's data releases than using our software, we host categories for science and data discussions.
For example, the DM System Science Team answers questions from the science community about LSST's planned data products.

To implement the forum, we chose the open source Discourse\footnote{\url{https://discourse.org}} platform because of its modern architecture and vibrant open source community.
Discourse is designed around a user trust system that largely eliminates the need for moderation (though the manual moderation tools are also excellent).
On the whole, Discourse is a low-maintenance system for DM to deploy and operate.

Prior to Discourse, we used mailing lists for a similar purpose.
The problem with mailing lists, though, is that their archives are largely undiscoverable, and their conversations are harder to link to.
Many DM team members initially favored retaining mailing lists because of the familiar workflow.
To accommodate those needs, we creating tooling to forward conversations from the Community forum to the old mailing lists as we deprecated them.
We continue to use that forwarding infrastructure, though Discourse's native email notification system is largely sufficient for newer users.
