\section{Model-Based Systems Engineering}

LSST uses Model-Based Systems Engineering\cite{2014SPIE.9150E..0MC}, with designs, requirements, and behavior described using the SysML language and stored within a database-backed tool, initially Enterprise Architect and now MagicDraw.
MagicDraw permits communication between teams through a unified model of the system.
It also allows for relatively easy maintenance of the requirements at all levels of the system as well as traceability between requirements and the system components that satisfy them.

Within the SysML language, there are specialized diagram types.
The Requirement Diagram documents requirements and their relationships.
Block Definition Diagrams highlighting interfaces and Internal Block Diagrams highlighting component breakdowns explicate the design of the system.
Activity Diagrams, Sequence Diagrams, State Machine Diagrams, and Use Case Diagrams show the system's intended behavior.
The MagicDraw tool allows these behavioral diagrams to be executed in a simulation mode to verify that the behavior is complete and correct.
All of this information is useful for the verification process.
Test cases are developed from requirements and behavior.

\subsection{Requirements}

Requirements are written using a combination of a formal specification, which uses verbs such as "shall" and "will" in a stylized manner, along with a description that explains the context and interpretation of the requirement.
Constraints are added to the requirement that document values of parameters used within the specification, along with their descriptions and units.
Each requirement is given a unique identifier generated based on its containing document and a monotonically-increasing sequence number.
This requirement id allows requirements to be safely referred to and linked together even if the containing document is reorganized or if a requirement is deleted or replaced by another.
Requirements document organization into sections is handled by grouping requirements into SysML packages.
Numbers prefixed to the titles of the packages and the requirements within them allow them to be sorted into a human-meaningful order.
These numbers are sequential within a containing package, not hierarchical, making renumbering and reorganization easier.
Requirements documents are generated using one of two custom macros, one for Microsoft Word documents and another for LaTeX documents.
The document generation macros sort the elements within packages and strip off the sequential prefix numbers, allowing the word processing tools to create the hierarchical section and requirement title numbers.
When LaTeX is used, the generated files are stored in git repositories and then follow the standard DM documentation release process.

\subsection{Requirements Documents}

Five levels of requirements documents are applicable to Data Management.
At the highest level, the LSST Science Requirements Document (SRD)\cite{LPM-17} sets out the overall scientific goals of the project in prose.
The LSST System Requirements\cite{LSE-29} are the project's formal response to the SRD, setting minimum, design, and "stretch" goals for requirement parameters.
The Observatory System Specifications\cite{LSE-30} documents requirements and budgets based on a high-level design, in particular partitioning the LSST system into Telescope \& Site, Camera, Data Management, and Education \& Public Outreach subsystems.
At this level, the Data Products Definition Document\cite{LSE-163} provides a full specification of the data products to be delivered by the LSST system.
The Data Management System Requirements (DMSR)\cite{LSE-61} are the flowdown to DM specifically.
At the same level as the DMSR, Interface Control Documents (ICDs) specify the relationships between the different subsystems.
Finally, within the DM system, component requirements documents are used where appropriate to constrain designs and provide for internal verification.

\subsection{Flowdown}

One of the excellent features of MagicDraw is the ability to rapidly document the flowdown of requirements.
Using a table with higher-level requirements on one side and lower-level requirements on the other, the "refines" relationship between the latter and the former can be created by merely toggling an arrow within appropriate table cells.
Unfortunately, it is more difficult to use the SysML "copy" relationship in cases where the lower-level requirement happens to be identical to the higher-level one.
It has been necessary to manually copy the requirement text (and generate a new requirement id, of course).

Since system components are also present in the MagicDraw model, the tool also helps with recording which requirements apply to each component, and, conversely, which components are needed to satisfy each requirement.

\subsection{Conclusion}

Ultimately, SysML and MagicDraw have been most useful for systems engineering purposes at the LSST system level, rather than within the Data Management system itself.
While the tool is fully capable of recording component designs in diagrammatic language, adoption has been somewhat difficult; developers tend to prefer to work directly on code.
In its place, more limited diagrams are produced using other tools such as OmniGraffle, Gliffy (particularly with the Confluence plugin), LaTeX, Archi, and \href{http://websequencediagrams.com}{websequencediagrams.com}.
